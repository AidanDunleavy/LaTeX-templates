%TeX command: LaTeX
%The first set of commands determines the overall look of your document.
\documentclass[12pt]{amsart}
\setlength{\textwidth}{6.1in}
\setlength{\oddsidemargin}{0pt}
\setlength{\evensidemargin}{0pt}
\setlength{\hoffset}{4pt}
\setlength{\parskip}{5pt}
\usepackage{amsmath}
\usepackage{amssymb}
\usepackage{amsthm}
\usepackage{setspace}

%The next two lines make it easy for me to put comments into your drafts.
\usepackage[usenames]{color}
\def\comment#1{{\color{red}{#1}}}

%The next two groups define "environments" useful for mathematics.
\theoremstyle{plain}
\newtheorem*{theorem}{Theorem}
\newtheorem*{corollary}{Corollary}
\newtheorem*{proposition}{Proposition}
\newtheorem*{lemma}{Lemma}

\theoremstyle{definition}
\newtheorem*{definition}{Definition}
\newtheorem*{construction}{Construction}
\newtheorem*{notation}{Notation}

%These are a couple of useful definitions.
\newcommand{\abs}[1]{\left\vert{#1}\right\vert}
\long\def\ignore#1\endignore{}
\newcommand{\id}{{\text{id}}}
\newcommand{\R}{{\mathbb{R}}}
\newcommand{\Z}{{\mathbb{Z}}}

\begin{document}

%Now we actually get to stuff that shows up in the document.
\title{Math 31500, Homework 0}

\author{Stu Dent}

\date{Jan.\ 0, 2020}

\maketitle

%The next line makes your document double spaced.
\doublespacing

\comment{\emph{This is where you put the statement of the problem that you're
solving.  You should recopy the problem \emph{exactly} as stated in 
the assignment.  For example, the problem might be the following, which
was an actual homework problem from a couple of years ago:}}

Prove that $f(x)=\displaystyle\frac{x}{\sqrt{1+x^2}}$ defines a bijection from $\R$
to $(-1,1)$ by finding its inverse function $g:(-1,1)\to\R$ and showing
that $g\circ f=\id_\R$ and $f\circ g=\id_{(-1,1)}$.

\comment{\emph{You should follow the statement of the problem with your solution.
Since this problem calls for a proof, you should
provide one.}}  

\begin{proof}
First, we note that $f(x)\in(-1,1)$, since 
\[
(f(x))^2=\frac{x^2}{1+x^2}=1-\frac{1}{1+x^2}<1,
\]
so taking square roots, we get
$\abs{f(x)}<1$, and therefore $ -1<f(x)<1$.

Next, let $g:(-1,1)\to\R$ be defined by
\[
g(x)=\frac{x}{\sqrt{1-x^2}}.
\]
Note that since $x\in(-1,1)$, $1-x^2>0$, so the function is well-defined.

Now we show that $g\circ f=\id_\R$.  Let $x\in\R$.  Then we have
\[
(g\circ f)(x)=g(f(x))=g\left(\frac{x}{\sqrt{1+x^2}}\right)
=\frac{\big(\frac{x}{\sqrt{1+x^2}}\big)}{\sqrt{1-\frac{x^2}{1+x^2}}}
=\frac{\big(\frac{x}{\sqrt{1+x^2}}\big)}{\sqrt{\frac{1}{1+x^2}}}=x.
\]

And finally, we show that $f\circ g=\id_{(-1,1)}$.  Let $x\in(-1,1)$.  Then we have
\[
(f\circ g)(x)=f(g(x))=f\left(\frac{x}{\sqrt{1-x^2}}\right)
=\frac{\big(\frac{x}{\sqrt{1-x^2}}\big)}{\sqrt{1+\frac{x^2}{1-x^2}}}
=\frac{\big(\frac{x}{\sqrt{1-x^2}}\big)}{\sqrt{\frac{1}{1-x^2}}}=x.
\]
Therefore $f:\R\to(-1,1)$ is a bijection with inverse $g$.
\end{proof}

\end{document}
